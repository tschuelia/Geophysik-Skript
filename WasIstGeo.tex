\chapter{Was ist Geophysik?}

Geophysik bedeutet "`Physik der Erde"' und beschäftigt sich mit den Eigenschaften und physikalischen Vorgängen in unserer Erde. Die Erforschung des Untergrunds erfolgt durch lokale geophysikalische Messungen an der Erdoberfläche. 

 Als Teilgebiete der Geowissenschaften und der Physik umfasst die Geophysik sehr viele Bereiche und hat viele Einsatzgebiete:
 
\vspace{0.7cm}
 
\begin{tabular}{ll}
  \textbf{Tiefenbereich} & \textbf{Anwendung}\\
  0 -- 50\,m & Ingenieurgeophysik (Baugrund, Grundwasser, Deponien \dots )\\
  < 5\,km & Exploration und Speicherung (Erdöl, Erdgas, $\text{CO}_2$, \dots )\\
  < 30 -- 40\,km & Tektonik (Plattentektonik, Krustenstruktur)\\
  < 6371\,km & Aufbau der globalen Erde
\end{tabular} 

In dieser Vorlesung werden wir verschiedene Messmethoden für geophysikalische Messungen zur Untersuchung in allen Tiefenbereichen kennen lernen.
 
\section{Arbeitsmethodik}
Wir wollen uns einmal anschauen, wie eine geophysikalische Untersuchung, unabhängig von der Messmethode, abläuft. \par

Am Anfang einer geophysikalischen Untersuchung steht eine Hypothese zur Struktur des Untergrundes. Anhand dieser Hypothese und mit Hilfe physikalischer Theorie lassen sich Messwerte vorhersagen. Diesen Prozess bezeichnet man als \textbf{Vorwärtsmodellierung}.

Der nächste Schritt ist die \textbf{Beobachtung}. Es wird eine geeignete Messmethode gewählt und durch Beobachtung der entsprechenden Messgröße an der Erdoberfläche werden Daten zur Struktur des Untergrundes gesammelt und anschließend bereinigt. 

Sind die Daten gesammelt und bereinigt, beginnt die Interpretation und \textbf{Inversion}. Durch systematisches Verändern der Hypothese des Untergrundmodells werden die hypothetischen Daten den tatsächlichen angepasst. Dabei ist zu beachten, dass es in der Regel mehrere Untergrundmodelle mit gleichen oder ähnlichen Messwerten gibt, was unter dem Begriff \textbf{Mehrdeutigkeit} bekannt ist. 

Am Ende dieses Prozesses steht ein physikalisches Modell des Untergrunds. 

Bei einem Großteil der Messungen lässt sich das Untersuchungsobjekt einfach isolieren, d.h. Lokalisierung und Größe sind leicht herauszufinden. Bei manchen Untersuchungsobjekten, wie z.B. dem Erdkern, ist dies allerdings nicht möglich. Man spricht dann von einer \textbf{komplexen Struktur}.   

